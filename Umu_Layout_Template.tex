% version 1.1 (Jan 2024)
% This LaTeX file is based on the original pdf template
% from the Informatics Department @ Umeå University

%%%%%%%%%%%%%%%%%%%%%%%%%%%% IMPORTANT %%%%%%%%%%%%%%%%%%%%%%%%%%%%%%%%%%%%%%%%
% In Overleaf: project settings (Menu) > change compiler: [LuaLaTex] 
% In VSCode: 1. adding this line in your VSCode user preferance.json file: 
%               "latex-workshop.latex.build.forceRecipeUsage": false, // allows magic comments
%            2. Add this to the top of your LaTex file (for LuaLaTeX): %!TEX program = lualatex
%%%%%%%%%%%%%%%%%%%%%%%%%%%%%%%%%%%%%%%%%%%%%%%%%%%%%%%%%%%%%%%%%%%%%%%%%%%%%%%%

% Enabling different LaTeX compilers in VS Code
% %!TEX program = lualatex % uncomment one "%" for enabling it

% Options for packages loaded elsewhere
\PassOptionsToPackage{unicode}{hyperref}
\PassOptionsToPackage{hyphens}{url}

\documentclass[]{article}
\usepackage{amsmath,amssymb}
\usepackage[style=apa]{biblatex}
\usepackage{iftex}
\usepackage{titlesec}
\usepackage{sectsty} %special header
\usepackage{fancyhdr} %special header

% Change all the margins to  2.5cm, headers from edge: 1.25cm, footer from edge: 1.25cm
\usepackage[margin=2.5cm, top=1.25cm, bottom=1.25cm, headheight=15pt]{geometry}

%%%%%%%%%%%%%%%%%% HEADING FORMAT %%%%%%%%%%%%%%%%%%%%%%

% Heading 1
\titleformat{\section}
  {\fontsize{16}{18}\bfseries}
  {\thesection}
  {24pt}
  {}
  [\titlespacing*{\section}{0pt}{24pt}{6pt}]
% Heading 2
\titleformat{\section}
  {\fontsize{14}{16}\bfseries}
  {\thesection}
  {12pt}
  {}
  [\titlespacing*{\section}{0pt}{12pt}{3pt}]
% Heading 3
\titleformat{\subsection}
  {\fontsize{12}{14}\bfseries}
  {\thesubsection}
  {12pt}
  {}
  [\titlespacing*{\subsection}{0pt}{12pt}{0pt}]

\ifPDFTeX
%%%%%%%%%%%%%%%% FONT -> Georgia %%%%%%%%%%%%%%%%%
  \usepackage{fontspec}
  \setmainfont{Georgia}
  
  \usepackage[utf8]{inputenc}
  \usepackage{textcomp} % provide euro and other symbols
\else % if luatex or xetex
  \usepackage{unicode-math} % this also loads fontspec
  \defaultfontfeatures{Scale=MatchLowercase}
  \defaultfontfeatures[\rmfamily]{Ligatures=TeX,Scale=1}
\fi
\usepackage{lmodern}
\ifPDFTeX\else
  % xetex/luatex font selection
\fi
% Use upquote if available, for straight quotes in verbatim environments
\IfFileExists{upquote.sty}{\usepackage{upquote}}{}
\IfFileExists{microtype.sty}{% use microtype if available
  \usepackage[]{microtype}
  \UseMicrotypeSet[protrusion]{basicmath} % disable protrusion for tt fonts
}{}
\makeatletter
\@ifundefined{KOMAClassName}{% if non-KOMA class
  \IfFileExists{parskip.sty}{%
    \usepackage{parskip}
  }{% else
    \setlength{\parindent}{0pt}
    \setlength{\parskip}{6pt plus 2pt minus 1pt}}
}{% if KOMA class
  \KOMAoptions{parskip=half}}
\makeatother
\usepackage{xcolor}
\usepackage{longtable,booktabs,array}
\usepackage{calc} % for calculating minipage widths
% Correct order of tables after \paragraph or \subparagraph
\usepackage{etoolbox}
\makeatletter
\patchcmd\longtable{\par}{\if@noskipsec\mbox{}\fi\par}{}{}
\makeatother
% Allow footnotes in longtable head/foot
\IfFileExists{footnotehyper.sty}{\usepackage{footnotehyper}}{\usepackage{footnote}}
\makesavenoteenv{longtable}
\ifLuaTeX
  \usepackage{luacolor}
  \usepackage[soul]{lua-ul}
\else
  \usepackage{soul}
\fi
\setlength{\emergencystretch}{3em} % prevent overfull lines
\providecommand{\tightlist}{%
  \setlength{\itemsep}{0pt}\setlength{\parskip}{0pt}}
\setcounter{secnumdepth}{-\maxdimen} % remove section numbering
\ifLuaTeX
  \usepackage{selnolig}  % disable illegal ligatures
\fi
\usepackage{bookmark}
\IfFileExists{xurl.sty}{\usepackage{xurl}}{} % add URL line breaks if available
\urlstyle{same}
\hypersetup{
  hidelinks,
  pdfcreator={LaTeX via pandoc}}

\author{}
\date{}


%%%%%%%%%%%%%%%%%%%%%%%%%%%%%%%%%%%%%%%%%%%%%%%%%%%%%%%%%%%%%%%%%%%%%%%%%%%%%%%%
%%%%%%%%%%%%%%%%%%%%%%%%%%%%%%%%%%%%%%%%%%%%%%%%%%%%%%%%%%%%%%%%%%%%%%%%%%%%%%%%


\begin{document}

\section*{\centering Abstract}

\emph{Your thesis should start with an abstract. The abstract should be
approximately 150 words long, written in Georgia 11 points, italics. It
must include the research question, basis for research, method, findings
and conclusions. That is, you want to tell the reader what you did, why
you did it, how you did it, what was the result of what you did, and how
does this contribute to research? An example abstract can be found in
Appendix 1. Beneath your abstract, you should list some keywords. These
give the reader a quick overview of the most important concepts in your
thesis.}

\textbf{Keywords}: Form, content, style, User Experience, Tangible,
music production

\section{1. General}\label{general}

The first chapter (1. General) should start immediately after the
abstract, not on a new page, and there should be no index in the thesis.
Master thesis work at the department of informatics is digitally
published in the DIVA system, administered by the University library, and
we want them to be consistent when it comes to format and style.

This document provides information concerning the required layout of the
thesis when published. Note that this document itself uses the layout
template it describes. Every word in this document belongs to a style, so
by clicking on different places you will see the style is used to
implement all features of the layout. Your thesis has a specific page
limit (20–25 pages for Master's 15 credits and Magister's 15 credits,
and 25–30 pages for Master's 30 credits, excluding references and
maximum 10 pages of appendices). Use them well, and remember that your
thesis should \ul{not} have an index!

\section{2. Headings (Heading 1)}\label{headings-heading-1}

Divide your article into clearly defined and numbered sections.
Subsections should be numbered 1.1 (then 1.1.1, 1.1.2, ...), 1.2, etc.
The abstract is not included in section numbering. Any section or
subsection should be given a brief heading. Each heading should appear
on its own separate line. Headings on different levels have different
layout (be consistent!). Section headings are written with 16 points,
bold. The spacing before should be 24 points and after 6 points. The
font in all headings should be Georgia.

\subsection{2.1 Heading 2}

Sublevel headings are 14 points bold. Spacing before should be 12
points, and after 3 points.

2.1.1 Heading 3

Third level headings should be 12 points bold. Spacing before 12 points,
and after 0 points. Try to avoid more sublevels. If you feel it is
absolutely necessary, do the next level like Heading 3, but not in bold.

\section{3. Layout}

\subsection{3.1 The body of the text}

The body of the text should be written using Georgia, 11 points. The
line spacing should be exactly 16 points. The margins should be
justified (i.e. stretched towards both margins).

New paragraphs are indented 0.5 cm in the first line, except the first
paragraph after a heading, quote or figure. There should not be empty
lines between paragraphs. Page numbers must be centred in the footer
and start on the same page as your abstract and introduction. All
margins should be set to 2.5 cm.

\subsection{3.2 The first page}

The first page of the thesis should start with an abstract. The purpose
of an abstract is to give the reader a quick overview of the major
points of the thesis. The abstract should be approximately 150 words
long, in Georgia 11 points, italics. The heading should be centred. An
example can be found in Appendix 1. After the abstract, you should list
some keywords central for your thesis. Then you move on to your
introduction.

\subsection{3.3 The title page}

The published thesis should have a cover page. That page has no page
number. The design of the cover is an Umeå University standard. Just
edit the template (another than this one) page by filling in your data
instead of the dummy data found in this template. Before publication,
your thesis will get a unique number in the department's publication
series. This should be put at the bottom of the title page. The course
administrator will provide that number when it is time for publication
of the thesis.

\subsection{3.4 Footnotes and references}

Footnotes should be used restrictively, numbered in sequence, and be
placed at the bottom of the page. The text in these should be Georgia,
10 points and the line spacing should be single\footnote{This is what a
  footnote looks like. Please try to use footnotes as little as
  possible.}. A footnote can for example be used for short
clarifications.

References in the text are done by stating the author(s) and year of
publication within parenthesis, for instance (Churchman, 1968) or
Churchman (1968) dependent upon if you want to use a strong or weak
referencing style. If the reference list contains more than one item by
the same author(s) from the same year, they are separated by adding a, b
and so on, to the year, for instance (Churchman, 1968b).

If there are more than two authors, you should list all authors the
first time the reference appears in the text (Jonsson, Westergren \&
Holmström, 2008) and the next time you only name the first author and
use et al. for the others (Jonsson et al., 2008). The full reference
should be provided in the reference list at the end of the thesis. The
layout for the reference list is shown in Appendix 2.

\subsection{3.5 Tables, figures, pictures}

For this kind of content there is no specific layout, just make sure it
is clear, easy to interpret and informative. All tables, figures and
pictures should be numbered, centered and have a text describing their
content. The number of the figure and the text should also be centered,
written in Georgia, 11 points, \emph{italics}, with 12 points spacing
before and 6 points after. The paragraph following the figure text
should have no indentation.

\subsection{3.6 Quotes}

If you are citing someone else's work, you must do this by enclosing the
cited text with quotation marks, followed by a reference including
author and page number. There are two main ways to do this:

\textbf{Example 1)} Mason (2002, p. 39) states: ``If your research is
valid, it means that you are observing, identifying or `measuring' what
you say you are''.

\textbf{Example 2)} On the subject of validity it has been said that
``If your research is valid, it means that you are observing,
identifying or `measuring' what you say you are'' (Mason, 2002, p. 39).

For longer quotes (more than 3 lines) you should use the specific
quotation style. It is the same as the body, except that the text is in
italics and the lines start and end 1 cm inside the normal margins.
Also, the line spacing is 15 points. A long quote should look like this:

``The fourth implication concerns the risk of going from stable to
rigid. The interlocked behaviour cycle was seen as ``solving'' the
problem of finding the necessary stability for organizational
operations. The problem of finding stability shifts in information
systems, such systems have different problems. In information systems,
the balance may tilt too far in the direction of stability. Instead,
avoiding rigidity becomes a problem'' (Nordström, 2003, p. 94).

\subsection{3.7 Appendix}

Note that as an appendix to the essay, there must also be a short
account - at most one page - of each student\textquotesingle s
contribution to the thesis project. There are no formal requirements for
this report that may be formulated freely. The reason why this appendix
should be included with the report is that students\textquotesingle{}
performance should be able to be assessed individually also in
connection with group work. Note that each student must contribute to
all assessment criteria for the thesis project, but within each
criterion you can of course divide work.

As the thesis has a page limit, you might want to consider placing some
images and tables (that are not crucial for understanding your
argumentation) in an appendix (as this is excluded from the page count).
All appendices should be numbered and placed at the end of the thesis,
after the reference list. The same layout instructions should be applied
to the text in the appendix. The only difference is that you must
include a header, like in the two appendices to this document. A maximum
of 10 pages can be used for appendices.

\section*{\centering Abstract}
% \section{Abstract}\label{abstract-1}

\emph{The open innovation model embraces the purposive flow of internal
and external ideas as a foundation for innovation and network formation.
While the open innovation paradigm has been successfully applied in
high-tech settings, there is a lack of research on adopters of open
innovation in other settings. We describe a case study conducted in a
process industry setting, focusing on the LKAB minerals group as it
makes a transition from a closed to a more open innovation context by
adopting remote diagnostics technology. This process has resulted in the
creation of new value networks. By tracing the reasoning behind the
organizational transformation and studying the technology used to carry
it through, we seek to explore the preconditions for open innovation and
provide insight into the role of IT in the process. Our findings show
that adoption of the open innovation model is grounded in developing
organizational environments that are conducive to innovation, including
expertise in creating a culture for knowledge sharing, building a
trustful environment, and a resourceful use of IT.}

\textbf{Keywords}: Open innovation, remote diagnostics, trust, value
networks

\section{1. Introduction}\label{introduction}

Contemporary industrial firms are under pressure. While they once took
pride in producing sought-after, superior quality products and exporting
them for profit, manufacturers are now being overtaken by firms located
in high-tech, low wage nations where products of comparable quality are
produced at much lower costs (Banker, Bardhan, Chang \& Lin, 2006;
Houseman, 2007). In order to regain its competitive advantage, industry
must re-invent itself by looking for alternative approaches to value
production. One approach is to improve the effectiveness and efficiency
of production, partly by organizational re-structuring, but also by
investing heavily in new information technology (IT) to develop
industrial processes. By using IT to monitor the production line, the
process can be speeded up and made more efficient, more streamlined, and
more cost-effective. To this end, the evolution in remote sensing
technology holds particular promise as it enables firms to monitor
complex processes from a distance.

Remote sensing has to date received some, but not sufficient, attention
within our field, for example through the work of Zuboff (1988), who
studied the use of smart machines within the pulp-and paper industry,
Puri (2007), with his work on the use of GIS in India, and Østerlie,
Almklov and Hepsø (2012) who studied undersea oil well maintenance.
These researchers have all shown how the use of IT creates possibilities
for knowledge generation. As industries now embrace the use of remote
sensing, they come into close contact with the technological potential
for boundary-spanning activities (Jonsson, Holmström \& Lyytinen, 2009),
hence providing the firms with an opportunity for both internal and
external process innovation and new ways of value creation.

\section{References (this is the heading you must use, and it should not
be
numbered)}\label{references-this-is-the-heading-you-must-use-and-it-should-not-be-numbered}

All of you should know that this is the general format for all of your
references in your reference list. The references must be listed in
alphabetical order. In this example, the references are listed in the
appendix. In your thesis, they should be listed after your final section.
For further examples of how to write your references, please see the
next page!

Berggren, U. and Bergkvist, T. (2006) Industrial Service Innovations --A
Disregarded Growth Engine. NUTEK, Report B2006:6, Available at:
http://pubikationer.tillvaxtverket.se, Accessed May 6, 2009.

Nilsson, K. (1989) Designing for Creativity - Toward a Theoretical Basis
for the Design of Interactive Information Systems. (UMADP-RRIPCS-8.89),
Institute of Information Processing, Umeå University, Umeå

Mason, J. (2002) \emph{Qualitative Researching}. London: Sage
Publications

Robey, D. (1981) Computer information systems and organization
structure. \emph{Communications of the ACM}, 24(10), pp. 679-687

Smith, A. (2012) New IT frightens teenagers. \emph{Daily News}, Jan. 23,
pp. 7-8

Walsham, G. (1997) Actor-network theory and IS research: Current status
and future prospects. In Lee, A.S., Libenau, J. \& J.I. DeGross (Eds.),
\emph{Information systems and qualitative research}, New York: Chapman
\& Hall, pp. 466-480.

Lientz, B. P. \& Swanson, E. B. (1980). \emph{Software Maintenance
Management}. Reading, Mass.: Addison-Wesley.

Westergren, U. H. (2007) Partnership Outsourcing Evolution -The Process
of Creating and Maintaining a Network of Actors. Paper presented at
\emph{the Fifteenth European Conference on Information Systems}, St.
Gallen, Switzerland, June 7-9.

\begin{longtable}[]{@{}
  >{\raggedright\arraybackslash}p{(\columnwidth - 2\tabcolsep) * \real{0.1492}}
  >{\raggedright\arraybackslash}p{(\columnwidth - 2\tabcolsep) * \real{0.8508}}@{}}
\toprule\noalign{}
\endhead
\bottomrule\noalign{}
\endlastfoot
For books & Surname, Initials (year) \emph{Title of Book}. Place of
Publication: Publisher

Mason, J. (2002) \emph{Qualitative Researching}. London: Sage
Publications \\
For book Chapters & Surname, Initials (year) Chapter title. In Editor's
Surname, Initials, (Ed(s).), \emph{Title of Book}, Place of Publication:
Publisher, Pages

Walsham, G. (1997) Actor-network theory and IS research: Current status
and future prospects. In Lee, A.S., Libenau, J. \& J.I. DeGross (Eds.),
\emph{Information systems and qualitative research}, New York: Chapman
\& Hall, pp. 466-480 \\
For journals & Surname, Initials (year) Title of article. \emph{Journal
Name}, Volume (Number), Pages

Robey, D. (1981) Computer information systems and organization
structure. \emph{Communications of the ACM}, 24(10), pp. 679-687 \\
For published conference proceedings & Surname, Initials (year) Title of
paper. In Surname, Initials (Ed.), \emph{Title of published proceeding
which may include place and date(s) held}, Place of Publication:
Publisher, Pages

Öbrand, L., Augustsson, N.-P., Holmström, J., Mathiassen, L. (2012) The
Emergence of Information Infrastructure Risk Management in IT Services.
In Sprague, R. H. Jr. (Ed.), \emph{Proceedings of 45\textsuperscript{th}
Annual Hawaii International Conference on System Sciences (HICSS)}, Los
Alamitos: IEEE Computer Society, pp. 4904-4913 \\
For unpublished conference proceedings & Surname, Initials (year) Title
of paper. Paper presented at \emph{Name of Conference}, Place of
Conference, Date of Conference

Westergren, U. H. (2007) Partnership Outsourcing Evolution -The Process
of Creating and Maintaining a Network of Actors. Paper presented at
\emph{the Fifteenth European Conference on Information Systems}, St.
Gallen, Switzerland, June 7-9 \\
For working papers & Surname, Intitials (year) Title of article, (Name
and number of Working paper series), Institution or organization, Place
of organization

Nilsson, K. (1989) Designing for Creativity - Toward a Theoretical Basis
for the Design of Interactive Information Systems. (UMADP-RRIPCS-8.89),
Institute of Information Processing, Umeå University, Umeå \\
For newspaper articles & Surname, Initials (year) Title of Article,
\emph{Newspaper}, Date, Pages

Smith, A. (2012) New IT frightens teenagers. \emph{Daily News}, Jan. 23,
pp. 7-8 \\
For electronic sources & If available online, the full URL should be
supplied at the end of the reference, as well as a date that the
resource was accessed.

Berggren, U. and Bergkvist, T. (2006) Industrial Service Innovations --A
Disregarded Growth Engine. NUTEK, Report B2006:6, Available at:
http://publikationer.tillvaxtverket.se, Accessed May 6, 2009. \\
\end{longtable}

\end{document}
